\documentclass{amsart}
\include{mydefs}

\newenvironment{lyxlist}[1]
{\begin{list}{}
{\settowidth{\labelwidth}{#1}
 \setlength{\leftmargin}{\labelwidth}
 \addtolength{\leftmargin}{\labelsep}
 \renewcommand{\makelabel}[1]{##1\hfil}}}
{\end{list}}

  \theoremstyle{remark}
  \newtheorem*{rem*}{\protect Remark}

\begin{document}

\section*{Math 312: Assignment 1}

\begin{lyxlist}{10.}
\item [{1.}]
We can expand $\binom{n}{k+1}$ into $\frac{n!}{(k+1)!(n-k-1)!}$ and $\binom{n}{k}$ into $\frac{n!}{(n-k)!}$

Then, the problem becomes:

= $\frac{n!}{(k+1)!(n-k-1)!}$ + $\frac{n!}{(n-k)!}$

= $\frac{n!(n-k)}{(k+1)!(n-k-1)!(n-k)}$ + $\frac{n!(k+1)}{(n-k)!(k+1)}$

For $0\leq k<n$ show that $\binom{n}{k+1}+\binom{n}{k}=\binom{n+1}{k+1}$
by a direct calculation.

\begin{lyxlist}{10.}
\item [{OPT:}] show that this holds even if $k\geq n$. \bigskip{}

\end{lyxlist}
\item [{2.}] For $n\geq0$ show that $\sum_{j=0}^{n}\binom{j}{0}=\binom{n+1}{1}$.\\
\emph{Hint:} once you unwind the definitions of both sides this is
not hard. \bigskip{}

\end{lyxlist}
\begin{center}
\textbf{Induction}
\par\end{center}

Use mathematical induction to prove the following assertions:
\begin{lyxlist}{10.}
\item [{3.}] Among every three consecutive positive integers there is one
that is divisible by $3$.\bigskip{}

\item [{4.}] Fix $k\geq0$ and show by induction on $n$ that for all $n\geq0$,
$\sum_{j=0}^{n}\binom{j}{k}=\binom{n+1}{k+1}$.\\
\emph{Hint:} For the induction step use problem 1.\bigskip{}

\item [{5.}] (Summation formulas) The case $k=1$ of problem 4 reads: $\sum_{j=0}^{n}j=\binom{n+1}{2}=\frac{n(n+1)}{2}$.
In this problem we will establish similar formulas for summing squares
and cubes of integers (you may recall these formulas from your integral
calculus course). Please express the formulas in the same form: a
product of terms linear in $n$ divided by an integer.

\begin{lyxlist}{10.}
\item [{(a)}] Show that $j^{2}=2\binom{j}{2}+\binom{j}{1}$. This means
that $\sum_{j=0}^{n}j^{2}=2\sum_{j=0}^{n}\binom{j}{2}+\sum_{j=0}^{n}\binom{j}{1}$
(why?). Use problem 4 to establish a formula for $\sum_{j=0}^{n}j^{2}$.
\item [{(b)}] Express $j^{3}$ as a combination of $\binom{j}{3},\,\binom{j}{2},\,\binom{j}{1}$
and use problem 4 to prove a formula for $\sum_{j=0}^{n}j^{3}$.
\item [{RMK}] You can check your formlas (but not your proofs) on the reverse
page.\bigskip{}

\end{lyxlist}
\item [{6.}] (Well-ordering) Use the well-ordering principle to show that
every amount of money payable with only nickels (10�) and quarters
(25�) is divisible by 5�.\bigskip{}

\end{lyxlist}
\begin{center}
\textbf{Factorials and primes}
\par\end{center}
\begin{lyxlist}{10.}
\item [{7.}] For $2\leq j\leq n$, show that $n!+j$ is not a prime number.
Conclude that there are arbitrarily long intervals containing no prime
numbers. \bigskip{}

\item [{8.}] For which prime numbers $p$ is $p+1$ also prime?\bigskip{}
\end{lyxlist}
\begin{rem*}
It is believed (the ``twin prime conjecture'') that there are infinitely
many primes $p$ for which $p+2$ is also prime. It is known (Polymath
8's refinement of Zhang's Theorem) that there is some $k$, $2\leq k\leq246$
such that $p,p+k$ are both prime infinitely often.
\end{rem*}
\newpage{}

Hint for problem 5:
\begin{eqnarray*}
\sum_{j=0}^{n}1 & = & n+1\\
\sum_{j=0}^{n}j & = & \frac{n(n+1)}{2}\\
\sum_{j=0}^{n}j^{2} & = & \frac{n(n+1)(2n+1)}{6}\\
\sum_{j=0}^{n}j^{3} & = & \left(\frac{n(n+1)}{2}\right)^{2}
\end{eqnarray*}

\end{document}
